\chapter{\ifenglish Introduction\else บทนำ\fi}

\section{\ifenglish Project rationale\else ที่มาของโครงงาน\fi}

การเรียนรู้เรื่องราวต่างๆเกี่ยวกับมหาวิทยาลัยนั้นทําได้หลายวิธี แต่เมื่อพูดถึงการใช้ชีวิตแล้วการรีวิวก็ถือว่า
เป็นการถ่ายทอดประสบการณ์ที่ดีวิธีหนึ่งนอกจากนี้ยังอาจเป็นแนวทางการใช้ชีวิตและช่วยในการแก้ไขปัญหา
หรือให้ข้อมูลข้อคิดเห็นและประสบการณ์เกี่ยวกับเรื่องราวที่คนกลุ่มหนึ่งให้ความสนใจร่วมกัน เช่น วิชา ร้านอาหาร เป็นต้น

การรีวิวสิ่งต่างๆที่มีประโยชน์กับนักศึกษามหาวิทยาลัยเชียงใหม่นั้นไม่ได้มีการรวบรวมไว้ ทําให้มีการกระ
จายไปในแพลตฟอร์มต่างๆ ประกอบกับแต่ละแพลตฟอร์มก็มีข้อจํากัดแตกต่างกันไป ทําให้การเข้าถึงรีวิวนั้น
ค่อนข้างยากต้องเสียเวลาพอสมควรบวกกับ การค้นหารีวิวต้องใช้การเรียบเรียงคีเวอร์สที่เหมาะสมเพื่อเพิ่ม
โอกาสในการเจอรีวิวที่เราต้องการ


ผู้พัฒนาได้มองเป็นปัญหาดังกล่าวจึงได้พัฒนาแอปพลิเคชันนี้ขึ้น เพื่อเพิ่มช่วยเสริมการเรียนรู้และเข้าใจ
การใช้ชีวิตในมหาวิทยาลัยแก่นักศึกษาและคนที่สนใจ รวมถึงการแก้ไขปัญหา ข้อสงสัยต่างๆ ผ่านการรีวิวที่ี
การแยกหมวดหมู่เพื่อเข้าถึงได้ง่ายต่อการค้นหาและการสร้างกระทู้สอบถามหรือขอความช่วยเหลือ

โดยทางผู้พัฒนาหวังว่าแอปพลิเคชันนี้จะช่วยในการให้ความรู้และช่วยในการประกอบการตัดสินใจต่างๆ
ได้ง่ายขึ้นผ่านการรีวิวและคําแนะนําจากคนที่มีประสบการณ์หรือมีความรู้ในเรื่องนั้นๆ

\section{\ifenglish Objectives\else วัตถุประสงค์ของโครงงาน\fi}
\begin{enumerate}
    \item เพื่อเป็นสื่อกลางในการถ่ายทอดการรีวิวต่างๆ
    \item เพื่อเป็นสื่อกลางในการไขข้อสงสัยต่างๆของมหาวิทยาลัย
    \item เป็นสื่อกลางในการติดต่อสื่อสารระหว่างบุคคลเพื่อขอคําแนะนําต่างๆ 
    \item เป็นสื่อกลางในการรวบรวมนักศึกษามหาวิทยาลัยเชียงใหม่และผู้ที่สนใจ
\end{enumerate}

\section{\ifenglish Project scope\else ขอบเขตของโครงงาน\fi}

\subsection{\ifenglish Hardware scope\else ขอบเขตด้านฮาร์ดแวร์\fi}
\begin{enumerate}
    \item รองรับบนโทรศัพท์มือถือแอนดรอยด์
\end{enumerate}
\subsection{\ifenglish Software scope\else ขอบเขตด้านซอฟต์แวร์\fi}
\begin{enumerate}
    \item สามารถเข้าใช้บริการโดยผ่านแฟตฟอร์มอื่นๆ ( facebook,google,email )
    \item สามารถเข้าไปอ่านรีวิว ค้นหารีวิวจากคําค้นหาหรือค้นหาโดยหมวดหมู่ รวมถึงการให้คะแนนรีวิว
    \item สามารถสร้างการรีวิว
    \item สามารถเข้าไปอ่านคําถามที่น่าสนใจ ตอบคําถาม รวมถึงการสร้างกระทู้สอบถามเพื่อไขข้อสงสัย
    \item สามารถติดต่อสื่อสารระหว่างบุคคลเพื่อขอคําแนะนําต่างๆ
\end{enumerate}
\section{\ifenglish Expected outcomes\else ประโยชน์ที่ได้รับ\fi}
\begin{enumerate}
    \item ช่วยในการประกอบการตัดสินใจต่างๆ
    \item ช่วยในการไขข้อสงสัยต่างๆ
    \item เป็นช่องทางในการติดต่อสื่อสารระหว่างบุคคล
\end{enumerate}
\section{\ifenglish Technology and tools\else เทคโนโลยีและเครื่องมือที่ใช้\fi}

\subsection{\ifenglish Hardware technology\else เทคโนโลยีด้านฮาร์ดแวร์\fi}
\begin{enumerate}
    \item โทรศัพท์แอนดรอยด์
\end{enumerate}
\subsection{\ifenglish Software technology\else เทคโนโลยีด้านซอฟต์แวร์\fi}
\begin{enumerate}
    \item Figma:สําหรับออกแบบตัวแอปพลิเคชัน  
    \item Android studio:พัฒนาแอนดรอยด์แอปพลิเคชัน
    \item Firebase:จัดการฐานข้อมูล การเข้าใช้บริการ
    \item app.diagrams.net:ออกแบบ user diagrams
    \item lucidchart: ออกแบบ schema database
    \item google cloud platform: เชื่อมต่อ google map api
    \item developers facebook: เชื่อมต่อ facebook api สำหรับการ login
\end{enumerate}
\section{\ifenglish Project plan\else แผนการดำเนินงาน\fi}

\begin{plan}{7}{2021}{1}{2022}
    \planitem{7}{2021}{8}{2021}{ศึกษาปัญหา ความต้องการ และรวบรวมข้อมูล}
    \planitem{7}{2021}{8}{2021}{ศึกษาภาษาและเครื่องมือที่ใช้ในการเขียนแอพพลิเคชัน}
    \planitem{8}{2021}{9}{2021}{กําหนดขอบเขตและวางแผนการดําเนินงาน}
    \planitem{8}{2021}{11}{2021}{ออกแบบ UX/UI}
    \planitem{8}{2021}{2}{2022}{พัฒนาแอพพลิเคชัน}
    \planitem{1}{2022}{2}{2022}{เขียนเอกสารรายงาน}
\end{plan}

% \section{\ifenglish Roles and responsibilities\else บทบาทและความรับผิดชอบ\fi}
% อธิบายว่าในการทำงาน นศ. มีการกำหนดบทบาทและแบ่งหน้าที่งานอย่างไรในการทำงาน จำเป็นต้องใช้ความรู้ใดในการทำงานบ้าง

% \section{\ifenglish%
% Impacts of this project on society, health, safety, legal, and cultural issues
% \else%
% ผลกระทบด้านสังคม สุขภาพ ความปลอดภัย กฎหมาย และวัฒนธรรม
% \fi}

% แนวทางและโยชน์ในการประยุกต์ใช้งานโครงงานกับงานในด้านอื่นๆ รวมถึงผลกระทบในด้านสังคมและสิ่งแวดล้อมจากการใช้ความรู้ทางวิศวกรรมที่ได้
