\chapter{\ifenglish Conclusions and Discussions\else บทสรุปและข้อเสนอแนะ\fi}

\section{\ifenglish Conclusions\else สรุปผล\fi}

จากการพัฒนาโครงงาน Wongwien ระบบที่ได้สามารถทำงานตามวัตถุประสงค์ของโครงงานได้นั่นคือได้แพลตฟอร์มที่สามารถใช้ใน
การรีวิวเรื่องต่างๆต่างหมวดหมู่ สามารถตั้งกระทู้เพื่อไขข้อสงสัยต่างๆ สามารถค้นหารีวิวได้อย่างรวดเร็ว สามารถสนทนาขอรับคำปรึกษาและ
สามารถเพิ่มวิธีการเข้าใช้งานระบบผ่านแฟตฟอร์มอื่นเพื่อความสะดวกสบาย

จากผลการทดสอบกับผู้ใช้งานสรุปได้ดังนี้
ระบบสามารถทำงานต่างๆได้ดีโดยรวม ปัญหาที่พบระหว่างการทดสอบความหน่วงที่เกิดจาการดึงข้อมูลมาแสดงผล ผู้ใช้งานสามารถใช้
งานแอพพลิเคชันโดยไม่เกิดความสับสน


\section{\ifenglish Challenges\else ปัญหาที่พบและแนวทางการแก้ไข\fi}

ในการทำโครงงานนี้ พบว่าเกิดปัญหาหลักๆ ดังนี้
\begin{enumerate}
    \item ปัญหาด้านการทดสอบระบบ อันเนื่องมาจากสถานการณ์โรคระบาดโควิด-19ทำให้การทดสอบระบบต้องทดสอบด้วยความล่าช้า
    เนื่องจากไม่ได้ทำเรื่องเบิกเงินเพื่อสมัครสมาชิกนำแอพขึ้น play store ทำให้ต้องใช้การติดตั้งแบบ apk ไฟล์ผ่าน android studio
    โดยต้องนัดเจอกันเพื่อติดตั้งแอพทดสอบ สถานการณ์โรคระบาดโควิด-19ทำให้การนัดเจอกันค่อนข้างยากขึ้น
    \item การทำงานของแอพพลิเคชันเมื่อมีอินเตอร์เน็ตที่ช้าทำให้การดึงข้อมูลช้าตามอาจสร้างความลำคาญกับผู้ใช้ได้
    \item ในการทำโปรเจคในครั้งนี้เรื่องของการแชร์ไปยังแอปพลิเคชันอื่นรอรับการแชร์ที่สมบูรณ์เมื่อเเชร์ผ่านอีเมล
    \item ระบบสามารถทำงานแบบออฟไลน์ได้ถ้าหากเคยมีการใช้งานมาก่อนจะเก็บเนื้อหาล่าสุดไว้
\end{enumerate}
\section{\ifenglish%
Suggestions and further improvements
\else%
ข้อเสนอแนะและแนวทางการพัฒนาต่อ
\fi
}

ข้อเสนอแนะเพื่อพัฒนาโครงงานนี้ต่อไป มีดังนี้
\begin{enumerate}
    \item ควรทำการพัฒนา UI ให้สวยงามน่าใช้งานมากกว่านี้และควรมีฟังชันอื่นๆที่น่าสนใจเพื่อดึงดูดผู้ใช้งาน
    \item ควรเพิ่มความสามารถในการปรับแต่งการรีวิวให้มันมีมากกว่านี้
\end{enumerate}
