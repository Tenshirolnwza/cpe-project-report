\maketitle
\makesignature

\ifproject
\begin{abstractTH}
% เขียนบทคัดย่อของโครงงานที่นี่
% การเขียนรายงานนี้เป็นส่วนหนึ่งของการทําโครงงานวิศวกรรมคอมพิวเตอร์ ได้ทำออกมาในรูปแบบของ Application โดยอุปกรณ์เน้นการทำงานบนโทรศัพท์มือถือ (android) เป็นหลัก ซึ่งโครงงานนี้จัดขึ้นเพื่อเป็นตัวเลือกหนึ่งที่ใช้ในการประกอบการตัดสินใจ หรือไขข้อสงสัย ผ่านระบบของการรีวิว กระดานกระทู้ และการปรึกษาพูดคุยเพื่อต้องการความช่วยเหลือ โดยแอพพลิเคชั่นสามารถรองรับการเข้าใช้งานโดยผ่านแพลตฟอร์มอื่น และสามารถช่วยค้นหาข้อมูลที่ต้องการได้ง่าย การใช้งานไม่ซับซ้อนสามารถเข้าใช้งานได้ไม่ยากเหมาะกับทุกเพศทุกวัย
\quad \quad \quad  การเขียนรายงานนี้เป็นส่วนหนึ่งของการทําโครงงานวิศวกรรมคอมพิวเตอร์ ได้ทำออกมาในรูปแบบของ Application โดย
อุปกรณ์เน้นการทำงานบนโทรศัพท์แอนดรอยด์เป็นหลัก ซึ่งโครงงานนี้จัดขึ้นเพื่อเป็นตัวเลือกหนึ่งที่ใช้ในการประกอบการตัดสินใจ ผ่าน
ระบบของการรีวิว ช่วยค้นหาข้อมูลประเภทรีวิว คำแนะนำจากการรีวิว ข้อเสนอแนะ สามารถปรึกษาพูดคุยเพื่อต้องการความช่วยเหลือ
โดยแอพพลิเคชั่นสามารถรองรับการเข้าใช้งานโดยผ่านแพลตฟอร์มอื่น และนอกจากนี้สามารถช่วยอำนวนความสะดวกในการเป็น
สื่อกลางในการสอบถามผ่านระบบตั้งกระทู้เพื่อให้ผู้ที่มีความรู้มาช่วยชี้นำ ตัวแอพพลิเคชันออกแบบมาสำหรับบุคคลทั่วไป
% การเขียนรายงานเป็นส่วนหนึ่งของการทำโครงงานวิศวกรรมคอมพิวเตอร์
% เพื่อทบทวนทฤษฎีที่เกี่ยวข้อง อธิบายขั้นตอนวิธีแก้ปัญหาเชิงวิศวกรรม และวิเคราะห์และสรุปผลการทดลองอุปกรณ์และระบบต่างๆ
% \enskip อย่างไรก็ดี การสร้างรูปเล่มรายงานให้ถูกรูปแบบนั้นเป็นขั้นตอนที่ยุ่งยาก
% แม้ว่าจะมีต้นแบบสำหรับใช้ในโปรแกรม Microsoft Word แล้วก็ตาม
% แต่นักศึกษาส่วนใหญ่ยังคงค้นพบว่าการใช้งานมีความซับซ้อน และเกิดความผิดพลาดในการจัดรูปแบบ กำหนดเลขหัวข้อ และสร้างสารบัญอยู่
% \enskip ภาควิชาวิศวกรรมคอมพิวเตอร์จึงได้จัดทำต้นแบบรูปเล่มรายงานโดยใช้ระบบจัดเตรียมเอกสาร
% \LaTeX{} เพื่อช่วยให้นักศึกษาเขียนรายงานได้อย่างสะดวกและรวดเร็วมากยิ่งขึ้น
\end{abstractTH}
\renewcommand{\baselinestretch}{2.0}

\begin{abstract}
    \quad \quad \quad  This project is a part of computer engineering project.This project presents an review application that
    support android phone.The objective of this project is helping decisions easily by a reviews system that
    support for finding review ,advice,consulting to another people.The application is decide to use easily, you can 
    loggin with another platform such as facebook and google.Moreover,this application has a space to support asking question for 
    people that know the answer to solve.
\end{abstract}

\iffalse
\begin{dedication}
This document is dedicated to all Chiang Mai University students.

Dedication page is optional.
\end{dedication}
\fi % \iffalse

\begin{acknowledgments}
    \quad \quad  โครงงานนี้จะไม่สำเร็จลุล่วงลงได้ ถ้าไม่ได้รับความกรุณาจาก ผศ.ดร.ลัชนา ระมิงค์วงศ์
    อาจารย์ที่ปรึกษา ที่ได้สละเวลาให้ความช่วยเหลือทั้งให้คำแนะนำ ให้ความรู้และแนวคิดต่างๆ รวมถึง
    รศ.ดร.ศักดิ์กษิต ระมิงค์วงศ์ และ ผศ.ดร.กําพล วรดิษฐ์ ที่ให้คำปรึกษาและคำแนะนำจนทำให้
    โครงงานเล่มนี้เสร็จสมบูรณ์ไปได้
    \newline

    ขอขบคุณทางภาควิชาภาควิชาวิศวกรรมคอมพิวเตอร์ คณะวิศวกรรมศาสตร์
มหาวิทยาลัยเชียงใหม่ ที่มีการให้งบประมาณสนับสนุนในการทำงาน
\newline

ขอขอบคุณเพื่อนๆ ที่ให้กำลังใจรวมถึงคำแนะนำที่ดีตลอดการทำโครงงานที่ผ่านมา
\newline

    นอกจากนี้ผู้จัดทำขอขอบพระคุณขอขอบพระคุณบิดา มารดาที่ได้ให้ชีวิต เลี้ยงดูสั่งสอน และ
ส่งเสียให้กระผมได้ศึกษาเล่าเรียนจนจบหลักสูตรปริญญาตรี หลักสูตรวิศวกรรมศาสตรบัณฑิต ซึ่งท่าน
ได้ให้กำลังใจ ในวันที่ท้อแท้ตลอดมา ซึ่งท่านยังเป็นแรงผลักดันให้กระผมสร้างสรรค์และมุ่งมั่นจนทำ
ให้โครงงานนี้สำเร็จ รวมทั้งขอขอบพระคุณอีกหลายๆท่านที่ไม่ได้เอ่ยนามมา ณ ที่นี้ ที่ได้ให้ความ
ช่วยเหลือตลอดมา หากหนังสือโครงงานเล่มนี้มีข้อผิดพลาดประการใด กระผมขอน้อมรับด้วยความ
ยินดี

    
% \texttt{acknowledgment} environment.

\acksign{2020}{5}{25}
\end{acknowledgments}%
\fi % \ifproject

\contentspage

\ifproject
\figurelistpage

\tablelistpage
\fi % \ifproject

% \abbrlist % this page is optional

% \symlist % this page is optional

% \preface % this section is optional
